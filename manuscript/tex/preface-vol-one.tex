\chapter{Prefazione}

{\centering
\emph{(Adattamento dal primo volume)}
\par}

\bigskip

\noindent
Da molto tempo, nei paesi buddhisti, è tradizione per i laici recarsi al
vicino monastero ogni luna nuova e luna piena per ascoltare un discorso
di Dhamma. In effetti, il Buddha stesso incoraggiò il suo Sangha a
mantenere questa pratica quindicinale. Anche se ora viviamo in un mondo
in cui le fasi della luna hanno meno significato, per molti è un aiuto a
rammentarsi dell'antica tradizione di cui facciamo parte.

Nel settembre del 2007 abbiamo iniziato a inviare versi del Dhammapada
scelti da `\emph{Dhammapada per la contemplazione}' del 2006. Veniva
offerta una strofa per ogni `giorno della luna', accompagnata da una
breve riflessione. Questo progetto è ora ben noto, si è diffuso
attraverso il passa-parola e le e-mail collettive. So che in varie parti
del mondo ci sono persone contente di ricevere un puntuale promemoria
dell'antica via, mentre affrontano la loro vita piena d'impegni. Altri
non vedono l'ora, ogni sera di luna nuova o di luna piena, quando
tornano dal lavoro, di aprire la casella di posta. Queste riflessioni
vengono usate in privato, riprodotte ampiamente, tradotte e fatte
girare. Ho anche sentito dire che fanno da base di discussione per gli
incontri dei gruppi settimanali di meditazione.

Era mia intenzione, condividendo in questo modo le mie personali
riflessioni, che magari anche altri si sentissero incoraggiati a mettere
in pratica la loro capacità riflessiva. C'è una tendenza, forse nei
praticanti buddhisti occidentali, a cercare di trovare la pace e la
comprensione fermando ogni forma di pensiero. Ma il Buddha ci dice che è
attraverso la `retta riflessione', che arriviamo a vedere la vera natura
della nostra mente; e non semplicemente fermando il pensiero.

Sono debitore nei confronti di molti che mi hanno aiutato nella
preparazione di questo materiale. Per i versi del Dhammapada ho
consultato varie autorevoli versioni. In particolare ho utilizzato il
lavoro del Venerabile Narada Thera (B.M.S. 1978), del Venerabile Ananda
Maitreya Thera (Lotsawa 1988), di Daw Mya Tin e gli editori della
Burmese Pitaka Association (1987) e di Ajahn Thanissaro. Per le storie
riportate insieme ai versi ho anche consultato
\href{http://www.tipitaka.net/}{www.tipitaka.net}.

\clearpage

Quando da più persone ho sentito dire che sarebbe stato utile
raccogliere in un libro queste riflessioni, mi sono rivolto al mio buon
amico Ron Lumsden. La sua notevole capacità di editing mi ha aiutato a
metter mano al mio lavoro per adattarlo a un numero maggiore di lettori.

Che le benedizioni che possono sorgere dalla compilazione di questo
volumetto siano condivise con tutti coloro che sono stati coinvolti
nella sua produzione e nel suo finanziamento. Che tutti quelli che
cercano la via possano trovarla e sperimentare alla fine la libertà. Che
tutti gli esseri possano cercare la via.

\bigskip

{\par\raggedleft
Bhikkhu Munindo\\
Monastero buddhista di Aruna Ratanagiri\\
Northumberland, Gran Bretagna\\
Stagione delle piogge (\emph{Vassa}) 2009
\par}
